% В этом файле следует писать текст работы, разбивая его на
% разделы (section), подразделы (subsection) и, если нужно,
% главы (chapter).

% Предварительно следует указать необходимую информацию
% в файле SETUP.tex

%% В этот файл не предполагается вносить изменения

% В этом файле следует указать информацию о себе
% и выполняемой работе.

\documentclass [fontsize=14pt, paper=a4, pagesize, DIV=calc]%
{scrartcl}
% ВНИМАНИЕ! Для использования глав поменять
% scrartcl на scrreprt

% Здесь ничего не менять
\usepackage [T2A] {fontenc}   % Кириллица в PDF файле
\usepackage [utf8] {inputenc} % Кодировка текста: utf-8
\usepackage [russian] {babel} % Переносы, лигатуры

%%%%%%%%%%%%%%%%%%%%%%%%%%%%%%%%%%%%%%%%%%%%%%%%%%%%%%%%%%%%%%%%%%%%%%%%
% Создание макроса управления элементами, специфичными
% для вида работы (курс., бак., маг.)
% Здесь ничего не менять:
\usepackage{ifthen}
\newcounter{worktype}
\newcommand{\typeOfWork}[1]
{
	\setcounter{worktype}{#1}
}

% ВНИМАНИЕ!
% Укажите тип работы: 0 - курсовая, 1 - бак., 2 - маг.,
% 3 - бакалаврская с главами.
\typeOfWork{1}
% Считается, что курсовая и бак. бьются на разделы (section) и
% подразделы (subsection), а маг. — на главы (chapter), разделы и
%  подразделы. Если хочется,
% чтобы бак. была с главами (например, если она большая),
% надо выбрать опцию 3.

% Если при выборе 2 или 3 вы забудете поменять класс
% документа на scrreprt (см. выше, в самом начале),
% то получите ошибку:
% ./aux/appearance.tex:52: Package scrbase Error: unknown option ` chapterprefix=

%%%%%%%%%%%%%%%%%%%%%%%%%%%%%%%%%%%%%%%%%%%%%%%%%%%%%%%%%%%%%%%%%%%%%%%%
% Информация об авторе и работе для титульной страницы

\usepackage {titling}

% Имя автора в именительном падеже (для маг.)
\newcommand {\me}{%
А.\,С.~Коненко
}

% Имя автора в родительном падеже (для курсовой и бак.)
\newcommand {\byme}{%
А.\,С.~Коненко%
}

% Научный руководитель
\newcommand{\supervisor}%
{к.ф.-м.н., старший преподаватель Р. Б. Штейнберг}

% идентифицируем пол (только для курсовой и бак.)
\newcommand{\bystudent}{
Студента %Студентки
}

% Год публикации
\date{2015}

% Название работы
\title{Оптимизация работы анализа псевдонимов в ОРС}

% Кафедра
%

\newcommand {\direction} {%
Направление подготовки\\02.\ifthenelse{\value{worktype} = 2}{04}{03}.02 ---
Фундаментальная информатика\\и информационные технологии%
}

%%%%%%%%%%%%%%%%%%%%%%%%%%%%%%%%%%%%%%%%%%%%%%%%%%%%%%%%%%%%%%%%%%%%%%%%
% Другие настраиваемые элементы текста

% Листинги с исходным кодом программ: укажите язык программирования
\usepackage{listings}
\lstset{
    language=[ISO]C++,%  Язык указать здесь
    basicstyle=\small\ttfamily,
    breaklines=true,%
    showstringspaces=false%
    inputencoding=utf8x%
}
% полный список языков, поддерживаемых данным пакетом, есть,
% например, здесь (стр. 13):
% ftp://ftp.tex.ac.uk/tex-archive/macros/latex/contrib/listings/listings.pdf

% Гиперссылки: настройте внешний вид ссылок
\usepackage%
[pdftex,unicode,pdfborder=0,draft=false,%backref=page,
    hidelinks, % убрать, если хочется видеть ссылки: это
               % удобно в PDF файле, но не должно появиться на печати
    bookmarks=true,bookmarksnumbered=false,bookmarksopen=false]%
{hyperref}

\usepackage {amsmath}      % Больше математики
\usepackage {amssymb}
\usepackage {textcase}     % Преобразование к верхнему регистру
\usepackage {indentfirst}  % Красная строка первого абзаца в разделе

\usepackage {fancyvrb}     % Листинги: определяем своё окружение Verb
\DefineVerbatimEnvironment% с уменьшенным шрифтом
	{Verb}{Verbatim}
	{fontsize=\small}

% Вставка рисунков
\usepackage {graphicx}

% Общее оформление
% ----------------------------------------------------------------
% Настройка внешнего вида

%%% Шрифты

% если закомментировать всё — консервативная гарнитура Computer Modern
\usepackage{paratype} % профессиональные свободные шрифты
%\usepackage {droid}  % неплохие свободные шрифты от Google
%\usepackage{mathptmx}
%\usepackage {mmasym}
%\usepackage {psfonts}
%\usepackage{lmodern}
%var1: lh additions for bold concrete fonts
%\usepackage{lh-t2axccr}
%var2: the package below could be covered with fd-files
%\usepackage{lh-t2accr}
%\usepackage {pscyr}

% Геометрия текста

\usepackage{setspace}       % Межстрочный интервал
\onehalfspacing

\newlength\MyIndent
\setlength\MyIndent{1.25cm}
\setlength{\parindent}{\MyIndent} % Абзацный отступ
\frenchspacing            % Отключение лишних отступов после точек
\KOMAoptions{%
    DIV=calc,         % Пересчёт геометрии
    numbers=endperiod % точки после номеров разделов
}

                            % Консервативный вариант:
%\usepackage                % ручное задание геометрии
%[%                         % (не рекомендуется в проф. типографии)
%  margin = 2.5cm,
  %includefoot,
  %footskip = 1cm
%] %
%  {geometry}

%%% Заголовки



\ifthenelse{\equal{\theworktype}{2}}{%
\KOMAoptions{%
    numbers=endperiod,% точки после номеров разделов
    headings=normal,   % размеры заголовков поменьше стандартных
    chapterprefix=true,% Печатать слово Глава в магистерской
    appendixprefix=true% Печатать слово Приложение
}
}

% шрифт для оформления глав и названия содержания
\newcommand{\SuperFont}{\Large\sffamily\bfseries}

% Заголовок главы
\ifthenelse{\value{worktype} > 1}{%
\renewcommand{\SuperFont}{\Large\normalfont\sffamily}
\newcommand{\CentSuperFont}{\centering\SuperFont}
\usepackage{fncychap}
\ChNameVar{\SuperFont}
\ChNumVar{\CentSuperFont}
\ChTitleVar{\CentSuperFont}
\ChNameUpperCase
\ChTitleUpperCase
}

% Заголовок (под)раздела с абзацного отступа
\addtokomafont{sectioning}{\hspace{\MyIndent}}

\renewcommand*{\captionformat}{~---~}
\renewcommand*{\figureformat}{Рисунок~\thefigure}

%%% Оглавление
\usepackage{tocloft}

% шрифт и положение заголовка
\ifthenelse{\value{worktype} > 1}{%
\renewcommand{\cfttoctitlefont}{\hfil\SuperFont\MakeUppercase}
}{
\renewcommand{\cfttoctitlefont}{\hfil\SuperFont}
}

% слово Глава
\usepackage{calc}
\ifthenelse{\value{worktype} > 1}{%
\renewcommand{\cftchappresnum}{Глава }
\addtolength{\cftchapnumwidth}{\widthof{Глава }}
}

% Очищаем оформление названий старших элементов в оглавлении
\ifthenelse{\value{worktype} > 1}{%
\renewcommand{\cftchapfont}{}
\renewcommand{\cftchappagefont}{}
}{
\renewcommand{\cftsecfont}{}
\renewcommand{\cftsecpagefont}{}
}

\ifthenelse{\value{worktype} > 1}{%
    \renewcommand{\cftchapaftersnum}{.}
}{
\renewcommand{\cftsecaftersnum}{.}
\renewcommand{\cftsubsecaftersnum}{.}
}
%%% Списки (enumitem)

\usepackage {enumitem}      % Списки с настройкой отступов
\setlist %
{ %
  leftmargin = \parindent, itemsep=.5ex, topsep=.4ex
} %

% По ГОСТу нумерация должны быть буквами: а, б...
%\makeatletter
%    \AddEnumerateCounter{\asbuk}{\@asbuk}{м)}
%\makeatother
%\renewcommand{\labelenumi}{\asbuk{enumi})}
%\renewcommand{\labelenumii}{\arabic{enumii})}

%%% Таблицы: выбрать более подходящие

\usepackage{booktabs} % считаются наиболее профессионально выполненными
%\usepackage{ltablex}
%\newcolumntype {L} {>{---}l}

%%% Библиография

\usepackage{csquotes}        % Оформление списка литературы
\usepackage[
  backend=biber,
  hyperref=auto,
  language=auto,
  citestyle=gost-numeric,
  bibstyle=gost-numeric,
]{biblatex}
\addbibresource{biblio.bib} % Файл с лит.источниками

% Настройка величины отступа в списке
\ifthenelse{\value{worktype} < 2}{%
\defbibenvironment{bibliography}
  {\list
     {\printtext[labelnumberwidth]{%
    \printfield{prefixnumber}%
    \printfield{labelnumber}}}
     {\setlength{\labelwidth}{\labelnumberwidth}%
      \setlength{\leftmargin}{\labelwidth}%
      \setlength{\labelsep}{\dimexpr\MyIndent-\labelwidth\relax}% <----- default is \biblabelsep
      \addtolength{\leftmargin}{\labelsep}%
      \setlength{\itemsep}{\bibitemsep}%
      \setlength{\parsep}{\bibparsep}}%
      \renewcommand*{\makelabel}[1]{\hss##1}}
  {\endlist}
  {\item}
}{}

% ----------------------------------------------------------------
% Настройка переносов и разрывов страниц

\binoppenalty = 10000      % Запрет переносов строк в формулах
\relpenalty = 10000        %

\sloppy                    % Не выходить за границы бокса
%\tolerance = 400          % или более точно
\clubpenalty = 10000       % Запрет разрывов страниц после первой
\widowpenalty = 10000      % и перед предпоследней строкой абзаца

% ----------------------------

% Стили для окружений типа Определение, Теорема...
% ���������� ������ (ntheorem)

\usepackage [thmmarks, amsmath] {ntheorem}
\theorempreskipamount 0.6cm

\theoremstyle {plain} %
\theoremheaderfont {\normalfont \bfseries} %
\theorembodyfont {\slshape} %
\theoremsymbol {\ensuremath {_\Box}} %
\theoremseparator {:} %
\newtheorem {mystatement} {�����������} [section] %
\newtheorem {mylemma} {�����} [section] %
\newtheorem {mycorollary} {���������} [section] %

\theoremstyle {nonumberplain} %
\theoremseparator {.} %
\theoremsymbol {\ensuremath {_\diamondsuit}} %
\newtheorem {mydefinition} {�����������} %

\theoremstyle {plain} %
\theoremheaderfont {\normalfont \bfseries} 
\theorembodyfont {\normalfont} 
%\theoremsymbol {\ensuremath {_\Box}} %
\theoremseparator {.} %
\newtheorem {mytask} {������} [section]%
\renewcommand{\themytask}{\arabic{mytask}}

\theoremheaderfont {\scshape} %
\theorembodyfont {\upshape} %
\theoremstyle {nonumberplain} %
\theoremseparator {} %
\theoremsymbol {\rule {1ex} {1ex}} %
\newtheorem {myproof} {��������������} %

\theorembodyfont {\upshape} %
%\theoremindent 0.5cm
\theoremstyle {nonumberbreak} \theoremseparator {\\} %
\theoremsymbol {\ensuremath {\ast}} %
\newtheorem {myexample} {������} %
\newtheorem {myexamples} {�������} %

\theoremheaderfont {\itshape} %
\theorembodyfont {\upshape} %
\theoremstyle {nonumberplain} %
\theoremseparator {:} %
\theoremsymbol {\ensuremath {_\triangle}} %
\newtheorem {myremark} {���������} %
\theoremstyle {nonumberbreak} %
\newtheorem {myremarks} {���������} %

% Титульный лист
% Макросы настройки титульной страницы
% В этот файл не предполагается вносить изменения

%\usepackage {showframe}

% Вертикальные отступы на титульной странице
\newcommand{\vgap}{\vspace{10pt}}
% \newcommand{\vgap}{\vspace{16pt}}

% Помещение города и даты в нижний колонтитул
\usepackage{scrlayer}
\DeclareNewLayer[
  foot,
  foreground,
  contents={%
    \raisebox{\dp\strutbox}[\layerheight][0pt]{%
      \parbox[b]{\layerwidth}{\centering Ростов-на-Дону\\ \thedate%
       \\\mbox{}
       }}%
  }
]{titlepage.foot.fg}
\DeclareNewPageStyleByLayers{titlepage}{titlepage.foot.fg}


\AtBeginDocument %
{ %
  %
  \begin{titlepage}
  %
    \thispagestyle{titlepage}

    {\centering
    %
    \MakeTextUppercase {МИНИСТЕРСТВО ОБРАЗОВАНИЯ И НАУКИ РФ}

    \vgap

    Федеральное государственное автономное образовательное\\
    учреждение высшего образования\\
    \MakeTextUppercase {Южный федеральный университет}

    \vgap

	Институт математики, механики и компьютерных наук
    имени~И.\,И.\,Воровича

    \vgap

    \direction

    \vspace* {\fill}

    \ifthenelse{\value{worktype} = 2}{%
    \me

    \vgap}{}

    \MakeTextUppercase{\thetitle}

    \ifthenelse{\value{worktype} = 2}{%
     \vgap

    Магистерская диссертация}{}
    \ifthenelse{\value{worktype} = 0}{
     \vgap

    Курсовая работа
    }{}%
    \ifthenelse{\value{worktype} = 1 \OR \value{worktype} = 3}{
     \vgap

    Выпускная квалификационная работа\\
    на степень бакалавра
    }{}%

    \vspace {\fill}

    \begin{flushright}
    \ifthenelse{\value{worktype} = 1 \OR \value{worktype} = 0}{
      \bystudent \ifthenelse{\value{worktype} = 0}{3}{4}\ курса\\
      \byme
    }{}

    \vgap

    Научный руководитель:\\
    \supervisor\\
    \ifthenelse{\value{worktype} = 2}{%
    Рецензент:\\
    ученая степень, ученое звание [, должность]
    И. О. Фамилия
    }{}
	\end{flushright}
\ifthenelse{\value{worktype} = 0}{
\vspace{\fill}
        \begin{flushleft}
          \begin{tabular}{cc}
            \underline{\hspace{4cm}}&\underline{\hspace{5cm}}\\
            {\small оценка (рейтинг)} & {\small  подпись руководителя}\\
          \end{tabular}
          \\[1cm]
%          Утверждаю:\\руководитель направления ФИИТ \underline{\hspace{4cm}}
%В.\,С.\,Пилиди
        \end{flushleft}
}{}
\ifthenelse{\value{worktype} = 1 \OR \value{worktype} = 3}{
\vspace{\fill}
        \begin{flushleft}
Допущено к защите:\\руководитель направления ФИИТ
\underline{\hspace{4cm}}
В.\,С.\,Пилиди
        \end{flushleft}
}{}


  	\vspace {\fill}

  %Ростов-на-Дону

    %\thedate

  }\end{titlepage}
  %
  %
  \tableofcontents
  %
  \clearpage
} %


% Команды для использования в тексте работы
% макросы для начала введения и заключения
\newcommand{\Intro}{\addsec{Введение}}
\ifthenelse{\value{worktype} > 1}{%
    \renewcommand{\Intro}{\addchap{Введение}}%
}

\newcommand{\Conc}{\addsec{Заключение}}
\ifthenelse{\value{worktype} > 1}{%
    \renewcommand{\Conc}{\addchap{Заключение}}%
}

% Правильные значки для нестрогих неравенств и пустого множества
\renewcommand {\le} {\leqslant}
\renewcommand {\ge} {\geqslant}
\renewcommand {\emptyset} {\varnothing}

% N ажурное: натуральные числа
\newcommand {\N} {\ensuremath{\mathbb N}}

% значок С++ — используйте команду \cpp
\newcommand{\cpp}{C\nolinebreak\hspace{-.05em}%
\raisebox{.2ex}{+}\nolinebreak\hspace{-.10em}%
\raisebox{.2ex}{+}}

% Неразрывный дефис, который допускает перенос внутри слов,
% типа жёлто-синий: нужно писать жёлто"/синий.
\makeatletter
    \defineshorthand[russian]{"/}{\mbox{-}\bbl@allowhyphens}
\makeatother

\endinput

% Конец файла


\begin{document}

\Intro

   Задачей настоящей бакалаврской работы является оптимизация работы анализа псевдонимов в ОРС.

\section{Определение}
\label{sec:examples}

\subsection{Что такое псевдонимы и зачем нужен анализ}

Два указателя a и b будем называть псевдонимами, если они указывают на одну и ту же область памяти. При этом мы можем это делать с некоторой уверенностью, если наш анализ делает упрощения и пренебрежения точностью.

\begin{lstlisting}
int* f(int* a, int* b)
{
    if (*a > 0)
    {
        *b = 0;
        return a;
    }
    else
    {
        *a = 0;
        return b;
    }
}

int main()
{
    int a = 1;
    int b = -1;

    int* c = f(&a, &b);
    int* d = f(&b, c);


    return 0;
}
\end{lstlisting}

В данном примере можно заметить, что если фактические параметры функции f окажутся псевдонимами, то ячейка памяти на которую они указывают будет обнулена. Ну и прочие любопытные замечания про этот наскоро написанный пример.

Целью анализа псевдонимов/алиасов является статическое (на этапе компиляции) определение для каждой пары указателей и, возможно, контекста вызова функции, как взаимоотносятся области памяти в которые указывают эти указатели. Всего выделяют четыре взаимоотношения:
\begin{description}
  \item[MustAlias] Указатели точно указывают в одну область памяти, т.е. являются псевдонимами.
  \item[PartialAlias] Объекты, на которые они указывают пересекаются в памяти, но начинаются не с одного адреса.
  \item[MayAlias] Указатели могут указывать в одну область памяти.
  \item[NoAlias] Указатели точно указывают в разные области памяти.
\end{description}

Результат анализа псевдонима имеет ряд применений:
\begin{itemize}
\item для уточнения результатов других анализов существующих в компиляторе.
\item для проверки условий применимости преобразований кода программы: некоторые преобразования невозможно сделать при псевдонимной связи между данными в блоке, а некоторые, наоборот, в своей работы опираются на наличие таких связей.
\item для статических проверок программы на возможные ошибки.
\end{itemize}

\subsection{Классификация анализов псевдонимов}

Анализ называют чувствительным к контексту (context-sensitive), если он различает вызовы анализируемой функции с различными взаимоотношениями между параметрами в смысле псевдонимов. В обратном случае, если для процедуры строится один контекстно-независимый результат, анализ называют не чувствительным к контексту (context-insensitive).

Анализ называют flow-sensitive, если он чувствителен к местам вхождений переданных указателей при анализе, иначе он является flow-insensitive.

Анализ называют чувствительным к полям структур (field-sensitive), если он способен различать поля структур при анализе. В обратном случае, если псевдонимом считается сразу вся структура, то анализ называется не чувствительным к полям структур (field-insensitive).

Если анализ основан на анализе типов (для языков вроде Haskell или Java) его называют  type-based. Если же анализ строится для языка, в котором есть сырые указатели и преобразования типов, то анализ строят на анализе потока управления программы и такой анализ называют flow-based.

\subsection{Известные подходы к анализу псевдонимов (на примере LLVM)}

\begin{enumerate}
\item NoAA - На все запросы отвечать "я не знаю".  
\item BasicAA - Баланс между скоростью и числом упрощений.
\item Steensgaard’s algorithm - межпроцедурный анализ с хорошей скоростью работы, но с большим числом упрощений.
\end{enumerate}

При использовании нескольких, анализы используются по цепочке, передавая свой результат следующему на вход.

\section{Существующая реализация анализа в ОРС}

Пользуясь описанной классификацией анализов существующую реализацию можно описать как flow-based, context-insensitive, flow-sensitive, field-sensitive (по умолчанию. Может быть и field-insensitive) анализ псевдонимов.

Существующая реализация достаточно точная, но весьма медленная с большим потреблением памяти.

\section{Ускорение анализа, параллельная реализация, чтобы нам сломать}

В существующей реализации было найдено два узких места:

\begin{itemize}
\item При большом ветвлении конец подпрограммы просматривается значительное число раз.
\item Подпрограмма анализируется при каждом её вызове.
\end{itemize}

Для обхода первого узкого места была усложнена логика обхода графа управления, которая теперь в местах ветвления пытается заглянуть вперед для нахождения более близкой точки объединения результатов анализа, нежели конец программы.

Для решения же второй проблемы была предпринята попытка воспользоватся идеями описанными в статье Лэм и Вильсона~\autocite{WilsonLamSIGPLAN95} и анализировать контексты для которых выполняются проверки, чтобы в случае вызова в одинаковых контекстах выполнять анализ однократно.

\Conc

Изменение обхода графа управления в процедуре не принесло большого выигрыша на основной массе тестов. Хорошо этот подход показал себя только на процедурах специального вида, где подряд много раз производится ветвление, а затем объединение.

\begin{lstlisting}
void f(int* a, int* b)
{
    if (...)
    {
        ...
    }
    else
    {
        ...
    }

    ...

    if (...)
    {
        ...
    }
    else
    {
        ...
    }
}
\end{lstlisting}

А вот применение идеи кэширования результатов анализа процедуры для контекстов дало выигрыш на всех достаточно сложных примерах.


% Печать списка литературы (библиографии)
\printbibliography[
    heading=bibintoc%
    ,title=Библиография % если хочется это слово
]
% Файл со списком литературы: biblio.bib
% Подробно по оформлению библиографии:
% см. документацию к пакету biblatex-gost
% http://ctan.mirrorcatalogs.com/macros/latex/exptl/biblatex-contrib/biblatex-gost/doc/biblatex-gost.pdf
% и огромное количество примеров там же:
% http://mirror.macomnet.net/pub/CTAN/macros/latex/contrib/biblatex-contrib/biblatex-gost/doc/biblatex-gost-examples.pdf

\end{document}